\documentclass[norsk,a4paper,12pt]{article}
\usepackage[utf8]{inputenc}
\usepackage{graphicx} %for å inkludere grafikk
\usepackage{verbatim} %for å inkludere filer med tegn LaTeX ikke liker
\usepackage{tabularx} %for tabeller
%\usepackage{booktabs} %for flott utforming av tabeller
\usepackage{amsmath}  %for mattetegn
%\usepackage{float}    %for a lase objekt
%\usepackage{listings} %Implementere kode i PDF
\usepackage{hyperref} %for URL referanse
%\usepackage{subfigure}%for multifigure

\title{INF5620 - Numerical methods for partial differential equations\\\vspace{2mm} \Large{Problem set 3}}
\author{\large Even Marius Nordhagen}
\date\today
\begin{document}

\maketitle

\begin{itemize}
\item For the Github repository containing programs and results, follow this link: 
\url{https://github.com/evenmn/master/tree/master/FYSKJM4480/Project1}
\end{itemize}

\section{Introduction}
The aim of this project is to study Hartree-Fock restricted and unrestricted. Many-particle system, the system is defined by the total Slater determinant $|\Phi\rangle=|\phi_1\phi_2\cdot\cdot\cdot\phi_N\rangle$ where $\phi_i$ is a single particle function (SPF). Electrons

\section{Theory}
For both RHF and UHF we can split up the wave function in a spin part and a position part:

\begin{equation}
\phi_{p,\sigma}(x)=\varphi_p(\vec{r})\chi_{\sigma}(s)
\end{equation}
where each orbit $\varphi$ is double occupied.
\begin{equation}
|\Phi_{RHF}\rangle=|\phi_{1\uparrow}\phi_{1\downarrow}\cdot\cdot\cdot\phi_{N\uparrow}\phi_{N\downarrow}\rangle=|\varphi_1\cdot\cdot\cdot\varphi_N\bar{\varphi}_1\cdot\cdot\cdot\bar{\varphi}_N\rangle
\end{equation}

\begin{equation}
\phi_{p,\sigma}(x)=\varphi_p^{\sigma}(\vec{r})\chi_{\sigma}(s)
\end{equation}
Total Slater determinant:
\begin{equation}
|\Phi_{RHF}\rangle=|\varphi_1^{\uparrow}\cdot\cdot\cdot\varphi_{N\uparrow}^{\uparrow}\bar{\varphi}_1^{\downarrow}\cdot\cdot\cdot\bar{\varphi}_{N\downarrow}^{\downarrow}\rangle
\end{equation}

Write down some general formulas here, Wave functions for restricted and unrestricted HF ex. 
\subsection{Energy expression}
The Hamiltonian of the system is assumed to on the form
\begin{equation}
\hat{H}=\hat{h}(i) + \hat{w}(i,j)
\end{equation}
where $\hat{h}$ is the kinetic energy operator and $\hat{w}$ is the potential energy. We want to find the total energy of the system, which is given by $\langle \Phi|\hat{H}|\Phi\rangle$. If we consider the kinetic energy, we can easily see that the kinetic energy of particle $i$ is given by $T_i=\langle\phi_i|\hat{h}|\phi_i\rangle$. For the potential it gets more complicated, since we have interaction between all the particles. We also need to make sure that the braket is anti-symmetric as we look at a system of fermions (electrons). For one electron pair $i$ and $j$ we therefore get the potential 
\begin{equation*}
V_{i,j}=\frac{1}{2}\langle\phi_i\phi_j|\hat{w}|\phi_i\phi_j-\phi_j\phi_i\rangle
\end{equation*}
with negative sign in the ket to get an anti-symmetric braket. Consider now a system of $N+1$ electrons, the total energy is given by
\begin{equation*}
\begin{split}
E=\langle\Phi|(\hat{h}(i) + \hat{w}(i,j))|\Phi\rangle
=\sum_{i=0}^N\langle\phi_i|\hat{h}|\phi_i\rangle+\frac{1}{2}\sum_{i=0}^N\sum_{\substack{j=0\\j\neq i}}^N\langle\phi_i\phi_j|\hat{w}|\phi_i\phi_j-\phi_j\phi_i\rangle.
\end{split}
\end{equation*}
We can apply this formula to find an expression for the UHF-energy. By plugging in the general UHF wave function from above, we obtain
\begin{align}
E_{UHF}&=\langle \Phi_{UHF}|\hat{H}|\Phi_{UHF}\rangle\\
&=\sum_{i,\sigma}\langle\varphi_i^{\sigma}|\hat{h}|\varphi_i^{\sigma}\rangle+\frac{1}{2}\sum_{i,j,\sigma,\tau}\langle\varphi_i^{\sigma}\varphi_j^{\tau}|\hat{w}|\varphi_i^{\sigma}\varphi_j^{\tau}-\varphi_j^{\tau}\varphi_i^{\sigma}\rangle\\
&=\sum_{i,\sigma}\langle\varphi_i^{\sigma}|\hat{h}|\varphi_i^{\sigma}\rangle+\frac{1}{2}\sum_{i,j,\sigma,\tau}\langle\varphi_i^{\sigma}\varphi_j^{\tau}|\hat{w}|\varphi_i^{\sigma}\varphi_j^{\tau}\rangle-\frac{1}{2}\sum_{i,j,\sigma}\langle\varphi_i^{\sigma}\varphi_j^{\sigma}|\hat{w}|\varphi_j^{\sigma}\varphi_i^{\sigma}\rangle.
\end{align}
Note that we do not have a sum over $\tau$ in the last term, but both $\tau$s are replaced by $\sigma$s. This is because the last term represents the exchange energy, and only particles with the same spin can exchange energy. Here we can also see why the anti-symmetric bra needs to have negative sign: fermions cause repulsive "exchange force", and with the standard positive force direction the exchange energy should be negative for fermions. 

\subsection{UHF equations}
...

\subsection{Roothan-Hall equations}
\subsection{Restricted Hartree-Fock}
With the restrictions $U=U^{\uparrow}=U^{\downarrow}$ and $N=N^{\uparrow}=N^{\downarrow}$ the UHF equations are still valid. The Roothan-Hall equation for RHF is
\begin{equation}
F(D)U=SU\epsilon
\end{equation}
where the Fock-operator is $F=h+J(D)-1/2K(D)$ and the density matrix reads $D=2U_{occ}U_{occ}^H$.
Similarly for UHF we have the Roothan-Hall equation
\begin{equation}
F^{\sigma}(D^{\uparrow},D^{\downarrow})U^{\sigma}=SU^{\sigma}\epsilon^{\sigma}
\end{equation}
with $F=h+J(D^{\uparrow}+D^{\downarrow})-K(D^{\sigma})$ and $D=U_{occ}^{\uparrow}(U_{occ}^{\uparrow})^H+U_{occ}^{\downarrow}(U_{occ}^{\downarrow})^H$. As we can see, RHF is like a more general version of UHF with $D^{\uparrow}=D^{\downarrow}=D$ and $U_{occ}^{\uparrow}=U_{occ}^{\downarrow}=U$, and a solution of RHF Roothan-Hall equations will therefore also be a solution of the UHF Roothan-Hall equations. 

%\begin{figure}[ht] 
%  \label{numvsclas} 
%  \begin{minipage}[b]{0.6\linewidth}
%    %\centering
%    \includegraphics[width=.9\linewidth]{figure_a.png} 
%    \caption{Method A\label{fa}} 
%    \vspace{4ex}
%  \end{minipage}%%
%  \begin{minipage}[b]{0.6\linewidth}
%    %\centering
%    \includegraphics[width=.9\linewidth]{figure_b.png} 
%    \caption{Method B\label{fb}} 
%    \vspace{4ex}
%  \end{minipage} 
%  \begin{minipage}[b]{0.6\linewidth}
%    %\centering
%    \includegraphics[width=.9\linewidth]{figure_c.png} 
%    \caption{Method C\label{fc}} 
%    \vspace{4ex}
%  \end{minipage}%% 
%  \begin{minipage}[b]{0.6\linewidth}
%    %\centering
%    \includegraphics[width=.9\linewidth]{figure_d.png} 
%    \caption{Method D\label{fd}} 
%    \vspace{4ex}
%  \end{minipage} 
%\end{figure}

%\begin{table} [H]
%\centering
%\caption{In this table you can find the mean error of the methods where $dt\in[0.001,0.%001,0.0001]$. $T=0.9$, all other parameters set to 1.}
%\begin{tabularx}{\textwidth}{XXXX} \hline
%\label{tab:error}
%Methods & $dt=0.001$ & $dt=0.0001$ & $dt=0.00001$ \\ \hline
%A & 7.207809809e-05 & 7.067358205e-05 & 7.053349755e-05 \\
%B & 0.0001344343021 & 0.0001427880242 & 0.0001434920618 \\
%C & 0.02330151372 & 0.02331403437 & 0.02331528300 \\
%D & 0.02330151372 & 0.02331403437 & 0.02331528300 \\ \hline
%\end{tabularx}
%\end{table}
%\begin{figure}[!htbp]
%\centering
%\includegraphics[width=100mm]{error_plot.png}
%\caption{This figure is connected to table (\ref{tab:error}) and shows the mean error %of the methods where $dt\in[0.001,0.001,0.0001]$. $T=0.9$, all other parameters set to %1. \label{error}}
%\end{figure}

\end{document}

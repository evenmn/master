\documentclass[norsk,a4paper,12pt]{article}
%\renewcommand{\thesection}{\Roman{section}} 
\usepackage[utf8]{inputenc}
\usepackage{graphicx} %for å inkludere grafikk
\usepackage{verbatim} %for å inkludere filer med tegn LaTeX ikke liker
\usepackage{tabularx}
\usepackage{booktabs}
\usepackage{amsmath}
\usepackage{float}
\usepackage{color}
\usepackage{listings}
\usepackage{hyperref}

\lstset{language=c++}
\lstset{basicstyle=\small}
\lstset{backgroundcolor=\color{white}}
\lstset{frame=single}
\lstset{stringstyle=\ttfamily}
\lstset{keywordstyle=\color{red}\bfseries}
\lstset{commentstyle=\itshape\color{blue}}
\lstset{showspaces=false}
\lstset{showstringspaces=false}
\lstset{showtabs=false}
\lstset{breaklines}
\lstset{postbreak=\raisebox{0ex}[0ex][0ex]{\ensuremath{\color{red}\hookrightarrow\space}}}
\usepackage{titlesec}

\setcounter{secnumdepth}{4}

\titleformat{\paragraph}
{\normalfont\normalsize\bfseries}{\theparagraph}{1em}{}
\titlespacing*{\paragraph}
{0pt}{3.25ex plus 1ex minus .2ex}{1.5ex plus .2ex}

\begin{document}

\begin{titlepage}
	\centering
	{\scshape\huge FYS4130 - Statistical Mechanics \par}
	\vspace{1cm}
	{\scshape\Large University of Oslo\par}
	\vspace{1.5cm}
	\includegraphics[width=0.25\textwidth]{uio.png}\par\vspace{1cm}
	{\Large\bfseries Obligatory assignment\par}
	\vspace{1.5cm}
	{\large\itshape Even Marius Nordhagen\par}
	\vfill
	{\large \today\par}
\end{titlepage}


\section{CLUSTER SIZE}
This problem aims an one dimensional lattice with $N$ sites at a given temperature. There is a atom at each site that can either have energy $+\epsilon$ or $-\epsilon$, and a chain of sites with energy $+\epsilon$ is called a cluster.

\subsection{Onebody probabilities}
Since each atom can be in two possible energy states, the one-particle partition function has two terms
\begin{equation}
Z_1=\sum_ie^{-\beta\epsilon_i}=e^{-\beta\epsilon}+e^{\beta\epsilon}
\end{equation}
and the probability of an atom having energy $+\epsilon$ and $-\epsilon$ is
\begin{equation}
P_+=\frac{e^{-\beta\epsilon}}{Z_1}=\frac{1}{1+e^{2\beta\epsilon}}\quad\text{and}\quad P_-=\frac{e^{\beta\epsilon}}{Z_1}=\frac{1}{1+e^{-2\beta\epsilon}}
\end{equation}
respectively.

\subsection{Probability of finding site in cluster}

\newpage
\section{ELECTRON GAS IN A MAGNETIC FIELD}
Here we study a ...

\subsection{}

\newpage
\section{QUANTUM COULOMB GAS}
We will look at a relativistic quantum gas of positive and negative charged particles with coulomb interaction. In addition we have an external box potential with walls of length $L$. Even though we cannot solve this analytically, we will get some results. The given Hamiltonian is as follows
\begin{equation}
{\cal H}=\sum_{i=1}^{2N}c|\boldsymbol{p}_i|+\sum_{i<j}^{2N}\frac{e_ie_j}{|\boldsymbol{r}_i - \boldsymbol{r}_j|}
\end{equation}

\subsection{Schrödinger equation}
The general Schrödinger equation for a many-body system is 
\begin{equation}
H\Psi_n=E_n\Psi_n
\end{equation}
where $\Psi_n$ is the total wave function of state $n$ and $E_n$ is the corresponding total energy. The wave functions are position dependent and the energies are dependent on the box size,
\begin{equation}
{\cal H}\Psi_n(\{\boldsymbol{r}_i\})=\epsilon_n(L)\Psi_n(\{\boldsymbol{r}_i\}).
\end{equation}
The exact wave functions are unknown, but there are some constraints that apply in general. Since the particles are indistinguishable, all observable should be the same although we swap two coordinates. This results in
\begin{equation}
P(a,b)=P(b,a)\quad\Rightarrow\quad |\Psi(a,b)|^2=|\Psi(b,a)|^2
\end{equation}
and
\begin{equation}
\Psi(a,b)=e^{i\phi}\Psi(b,a).
\end{equation}
Swapping twice must give back the initial wave function, so there are two possible choices of $\phi$: 0 and $2\pi$. The first one gives a symmetric total wavefunction under exchange of two particles, which is antisymmetric for the second choice. We denote the former as bosons and the latter as fermions, where the Pauli principle is a consequence of antisymmetry.

\subsection{Scaling}
We will now scale the system with respect to $L$ such that the box gets size $1^d$ where $d$ is the number of dimensions. 

\newpage
\section{NON-INTERACTING FERMIONS}
In this problem we study a non-interacting fermi gas with chemical potential $\mu$. The particle states are given by the vector $\vec{k}$, which is $d$-dimensional where $d$ is number of spatial dimensions, and the corresponding energy is $\epsilon(\vec{k})$.

\subsection{Joint probability}
The joint probability of finding a particle in a state given by the occupation numbers $\{n_{\vec{k}}\}$ is given by the general probability statement in a grand canonical ensemble
\begin{equation}
P(\{n_{k}\})=\frac{e^{-\beta n_k(\epsilon_k-\mu)}}{\Xi}
\end{equation}
where $\Xi$ is the grand canonical partition function, given by
\begin{equation}
\Xi
\end{equation}

\subsection{}

\subsection{Maximum entropy of random variable}
A random variable has $l$ discrete outcomes with their own respective probabilities. Since the probabilities are independent, we can find the information entropy
\begin{equation}
S=H(p_1,\hdots,p_n)=-\sum_{n=1}^lp_n\log p_n
\end{equation}
where $p_i$ is the probability of outcome $i$. We cannot find the exact entropy without knowing the probabilities, but we can find which probabilities that give maximum entropy.

\subsection{Entropy of fermi gas}

\iffalse
\begin{figure}[!htbp]
\centering
\includegraphics[width=100mm]{example}
\caption{Time evolution of the motion in y-direction, where both the numerical and classical solution are plotted. Plotted over 6 periods and 600 timesteps per period. \label{y_time}}
\end{figure}
\fi
\end{document}

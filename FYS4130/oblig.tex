\documentclass[norsk,a4paper,12pt]{article}
\renewcommand{\thesection}{\Roman{section}} 
\usepackage[utf8]{inputenc}
\usepackage{graphicx} %for å inkludere grafikk
\usepackage{verbatim} %for å inkludere filer med tegn LaTeX ikke liker
\usepackage{tabularx}
\usepackage{booktabs}
\usepackage{amsmath}
\usepackage{float}
\usepackage{color}
\usepackage{listings}
\usepackage{hyperref}

\lstset{language=c++}
\lstset{basicstyle=\small}
\lstset{backgroundcolor=\color{white}}
\lstset{frame=single}
\lstset{stringstyle=\ttfamily}
\lstset{keywordstyle=\color{red}\bfseries}
\lstset{commentstyle=\itshape\color{blue}}
\lstset{showspaces=false}
\lstset{showstringspaces=false}
\lstset{showtabs=false}
\lstset{breaklines}
\lstset{postbreak=\raisebox{0ex}[0ex][0ex]{\ensuremath{\color{red}\hookrightarrow\space}}}
\usepackage{titlesec}

\setcounter{secnumdepth}{4}

\titleformat{\paragraph}
{\normalfont\normalsize\bfseries}{\theparagraph}{1em}{}
\titlespacing*{\paragraph}
{0pt}{3.25ex plus 1ex minus .2ex}{1.5ex plus .2ex}

\begin{document}

\begin{titlepage}
	\centering
	{\scshape\huge FYS4130 - Statistical Mechanics \par}
	\vspace{1cm}
	{\scshape\Large University of Oslo\par}
	\vspace{1.5cm}
	\includegraphics[width=0.25\textwidth]{uio.png}\par\vspace{1cm}
	{\Large\bfseries Obligatory assignment\par}
	\vspace{1.5cm}
	{\large\itshape Even Marius Nordhagen\par}
	\vfill
	{\large \today\par}
\end{titlepage}


\section{CLUSTER SIZE}
This problem aims an one dimensional lattice with $N$ sites at a given temperature. There is a atom at each site that can either have energy $+\epsilon$ or $-\epsilon$, and a chain of sites with energy $+\epsilon$ is called a cluster.

\subsection{Onebody probabilities}
Since each atom can be in two possible energy states, the one-particle partition function has two terms
\begin{equation}
Z_1=\sum_ie^{-\beta\epsilon_i}=e^{-\beta\epsilon}+e^{\beta\epsilon}
\end{equation}
and the probability of an atom having energy $+\epsilon$ and $-\epsilon$ is
\begin{equation}
P_+=\frac{e^{-\beta\epsilon}}{Z_1}=\frac{1}{1+e^{2\beta\epsilon}}\quad\text{and}\quad P_-=\frac{e^{\beta\epsilon}}{Z_1}=\frac{1}{1+e^{-2\beta\epsilon}}
\end{equation}
respectively.

\subsection{Probability of finding site in cluster}


\section{ELECTRON GAS IN A MAGNETIC FIELD}
Here we study a ...

\subsection{}


\section{QUANTUM COULOMB GAS}
Relativistic quantum gas with coulomb interaction

\subsection{}


\section{NON-INTERACTING FERMIONS}
In this problem we study a non-interacting fermi gas with chemical potential $\mu$. The particle states are given by the vector $\vec{k}$, which is $d$-dimensional where $d$ is number of spatial dimensions, and the corresponding energy is $\epsilon(\vec{k})$.

\subsection{Joint probability}
The joint probability of finding a particle in a state given by the occupation numbers $\{n_{\vec{k}}\}$ is given by the general probability statement in a grand canonical ensemble
\begin{equation}
P(\{n_{k}\})=\frac{e^{-\beta n_k(\epsilon_k-\mu)}}{\Xi}
\end{equation}

\iffalse
\begin{figure}[!htbp]
\centering
\includegraphics[width=100mm]{example}
\caption{Time evolution of the motion in y-direction, where both the numerical and classical solution are plotted. Plotted over 6 periods and 600 timesteps per period. \label{y_time}}
\end{figure}
\fi
\end{document}
